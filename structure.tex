%%%%%%%%%%%%%%%%%%%%%%%%%%%%%%%%%%%%%%%%%
% This template originates from:
% http://www.LaTeXTemplates.com
%%%%%%%%%%%%%%%%%%%%%%%%%%%%%%%%%%%%%%%%%

%----------------------------------------------------------------------------------------
%	PACKAGES AND OTHER DOCUMENT CONFIGURATIONS
%----------------------------------------------------------------------------------------
\usepackage{xcolor}
\usepackage{amsmath, amsfonts, amsthm} % Math packages
\usepackage{minted}
\usepackage{listings} % Code listings, with syntax highlighting
% Umbenennung von Listing auf Auflistung
\renewcommand\listoflistingscaption{Auflistungen}
\usepackage{caption}
\DeclareCaptionLabelFormat{Auflistung}{#1 #2}
\captionsetup[listing]{labelformat=Auflistung, name=Auflistung}


\usepackage{pdflscape}
\usepackage{float} % Enables [H] positioning option for figures

\usepackage[ngerman]{babel} % English language hyphenation
\usepackage{tabularx}
\usepackage{graphicx} % Required for inserting images
\usepackage{hyperref} % for hyperlinks
\providecommand*{\listingautorefname}{Auflistung}
\graphicspath{{Figures/}{./}} % Specifies where to look for included images (trailing slash required)

\usepackage{booktabs} % Required for better horizontal rules in tables

\numberwithin{equation}{section} % Number equations within sections (i.e. 1.1, 1.2, 2.1, 2.2 instead of 1, 2, 3, 4)
\numberwithin{figure}{section} % Number figures within sections (i.e. 1.1, 1.2, 2.1, 2.2 instead of 1, 2, 3, 4)
\numberwithin{table}{section} % Number tables within sections (i.e. 1.1, 1.2, 2.1, 2.2 instead of 1, 2, 3, 4)
\numberwithin{listing}{section} % Number tables within sections (i.e. 1.1, 1.2, 2.1, 2.2 instead of 1, 2, 3, 4)

\setlength\parindent{0pt} % Removes all indentation from paragraphs

\usepackage{enumitem} % Required for list customisation
\setlist{noitemsep} % No spacing between list items

\usepackage{color}

\definecolor{pblue}{rgb}{0.13,0.13,1}
\definecolor{pgreen}{rgb}{0.0,0.5,0}
\definecolor{pred}{rgb}{0.9,0,0}
\definecolor{pgrey}{rgb}{0.46,0.45,48}

%----------------------------------------------------------------------------------------
%	DOCUMENT MARGINS
%----------------------------------------------------------------------------------------

\usepackage{geometry} % Required for adjusting page dimensions and margins

\geometry{
	paper=a4paper, % Paper size, change to letterpaper for US letter size
	top=2.5cm, % Top margin
	bottom=3cm, % Bottom margin
	left=3cm, % Left margin
	right=3cm, % Right margin
	headheight=0.75cm, % Header height
	footskip=1.5cm, % Space from the bottom margin to the baseline of the footer
	headsep=0.75cm, % Space from the top margin to the baseline of the header
	%showframe, % Uncomment to show how the type block is set on the page
}

%----------------------------------------------------------------------------------------
%	FONTS
%----------------------------------------------------------------------------------------

\usepackage[utf8]{inputenc} % Required for inputting international characters
\usepackage[T1]{fontenc} % Use 8-bit encoding

\usepackage{fourier} % Use the Adobe Utopia font for the document

%----------------------------------------------------------------------------------------
%	SECTION TITLES
%----------------------------------------------------------------------------------------

\usepackage{sectsty} % Allows customising section commands

\sectionfont{\vspace{6pt}\normalfont\scshape} % \section{} styling
\subsectionfont{\normalfont\bfseries} % \subsection{} styling
\subsubsectionfont{\normalfont\itshape} % \subsubsection{} styling
\paragraphfont{\normalfont\scshape} % \paragraph{} styling

\usepackage[titles]{tocloft} %customizable table of content

%----------------------------------------------------------------------------------------
%	Hyper Links Setup - Definition of the different links styles
%----------------------------------------------------------------------------------------
\hypersetup{
    colorlinks=true,
    linkcolor=black,
    filecolor=magenta,      
    urlcolor=cyan,
    citecolor=blue,
}

%----------------------------------------------------------------------------------------
%	HEADERS AND FOOTERS
%----------------------------------------------------------------------------------------
\usepackage{fancyhdr}% Required for customising headers and footers
\usepackage[lastpage, user]{zref}
\fancyhf{}
\lhead{HSLU Datenbanksysteme} % Left header
\rhead{Team 4} % right header
%\lfoot{\today} % left footer %% Comment Oli: M.M nach nicht relevant auf jeder Seite
%\rfoot{\thepage/\zpageref{LastPage}} % right footer
\renewcommand{\headrulewidth}{0.5pt}
\renewcommand{\footrulewidth}{0.5pt}
%\usepackage{scrlayer-scrpage} % Required for customising headers and footers

%\ohead*{} % Right header
%\ihead*{} % Left header
%\chead*{} % Centre header

%\ofoot*{} % Right footer
%\ifoot*{} % Left footer
%\cfoot*{\pagemark} % Centre footer

%----------------------------------------------------------------------------------------
%	JAVA CODE STYLE
%----------------------------------------------------------------------------------------

\lstdefinestyle{javaCodeStyle}{
	language=Java, % Use Java functions/syntax highlighting
	frame=single, % Frame around the code listing
	showspaces=false,
	showtabs=false,
	breakatwhitespace=true,
	breaklines=true,
	postbreak=\mbox{\textcolor{red}{$\hookrightarrow$}\space},
	commentstyle=\color{pgreen},
	keywordstyle=\color{pblue},
	stringstyle=\color{pred},	
	basicstyle=\scriptsize,
	showstringspaces=false, % Don't put marks in string spaces
	numbers=left, % Line numbers on left
	numberstyle=\tiny, % Line numbers styling
}

\lstset{style=javaCodeStyle}


\usepackage{minted}
%----------------------------------------------------------------------------------------
%	New Commands
%----------------------------------------------------------------------------------------
%Define texttt with hyphenation enabled
\DeclareTextFontCommand{\mytexttt}{\ttfamily\hyphenchar\font=45\relax}

%For code in text
\newcommand{\code}[1]{\mytexttt{#1}}

%Prevent dots in TOC
\renewcommand{\cftdot}{}

%SubItem for Itemize
\newcommand{\subItem}[1]{
    {\setlength\itemindent{15pt} \item[-] #1}
}


\usepackage[style=numeric,sorting=none]{biblatex}
\usepackage{csquotes}

\usepackage{tcolorbox}
